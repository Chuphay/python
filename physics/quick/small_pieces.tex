\documentclass {article}
\usepackage{graphicx}
\usepackage{amsmath}



\begin{document}


\section{Meta Theory}


Who, What, Where, When, Why, How?
\\[0.15in]
what?    A new theory of  gravity
\\
why?   100 years quantumgravity = GR + QM . Other proposals
\\[1in]

Moving Goal Posts
\\

MVP
\\

What is a theory?
\\[1in]

Hypothesis vs. Postulate vs. Assumptions? (Main Hypothesis with mathematical framework postulates?)
\\[1in]



What does a theory of gravitation need?
\\[2in]

What would be nice to have? 
\begin{itemize}
	\item Simplicity : Occam's razor
	\item A wow!
\end{itemize}

\vspace{10pt}





\newpage


\section{Maxwell Equations}
\subsection{From Electromagnetism to Gravity-magnetism}

We assume that gravity is correctly described by Maxwell's Equations:

$$\nabla \times \vec B  - \partial_t \vec E  = -4 \pi G \vec J ~~~~~~~~ \nabla \times \vec E + \partial_t \vec B = 0    ~~~~~~~~~~~ (1)$$

$$\nabla \cdot \vec E = -4 \pi G \rho ~~~~~~~~~~ \nabla \cdot \vec B = 0   ~~~~~~~~~~~ (2)$$

and, for completeness, the conservation of mass-charge:

$$\nabla \cdot \vec J + \partial_t \rho = 0 ~~~~~~~~~~~ (3)$$

Where we are in units where c = 1. The constant G differs in different systems of units. In SI units, it is equal to $6.674 \times 10^{-11} \frac {N \cdot m^2}{kg^2}$ 

\newpage 
\subsection{Solenoidal and Irrotational Vectors}
While this step is not strictly necessary, and all of the results in the rest of the paper can still be derived, it actually tends to make the math a little cleaner to split the Maxwell equations into their Irrotational and Solenoidal forms.

A vector $\vec V$ can be split into a solenoidal vector $\vec V_S$ and an irrotational vector $\vec V_I$:

$$\vec V = \vec V_I + \vec V_S $$

where

$$\nabla \cdot \vec V_S = 0 ~~~~~ \nabla \times \vec V_I = 0 $$

Using this notation we can rewrite the Maxwell equations into irrotational equations:

$$\nabla \cdot \vec E_I = -4 \pi G  \rho$$

$$\nabla \cdot \vec J_I + \partial_t \rho = 0 $$

$$ \partial_t \vec E_I = 4 \pi G \vec J_I $$

And solenoidal equations:

$$\nabla \times \vec E_S + \partial_t \vec B_S = 0 $$

$$\nabla \times \vec B_S - \partial_t \vec E_S =  - 4 \pi G \vec J_S$$

The thing we really want to say, and I'm not sure if this is an assumption or is just totally obvious, is that, at least when we are in the rest frame of the mass (which is the frame of reference that we will be in for the rest of this paper), these equations de-couple. We have one set of equations that describe the forces due to the mass and another set of equations that describe radiation. We can make this explicit by defining a new current:

$$\vec {J_S'} = - G \vec {J_S}$$

which changes the inhomogenous Solenoidal equation into:

$$\nabla \times \vec B_S - \partial_t \vec E_S =  4 \pi \vec {J_S'} $$

but affects none of the other equations at all!

Because this form of the equations is the same as that presented in books on radiation. I will get rid of the prime on the Solenoidal current, and use the following equations as the solenoidal (radiation) Maxwell equations:

$$\nabla \times \vec B_S - \partial_t \vec E_S =  4 \pi \vec J_S$$

$$\nabla \times \vec E_S + \partial_t \vec B_S = 0 $$
\newpage

\section{Solutions to the Maxwell Equations}
\subsection{Spherical Vectors}
Following Berrara (http://iopscience.iop.org/article/10.1088/0143-0807/6/4/014/meta) we assume that any vector field, $\vec V$, and any scalar, $g$, can be expanded as follows:

$$\vec V (\vec x) = \sum_{l=0}^{\infty} \sum_{m=-l}^{l} \left[a_{lm}(r) \vec Y_{lm}(\theta, \phi) +b_{lm}(r) \left( \hat r \times \vec X_{lm} (\theta, \phi) \right) +c_{lm}(r) \vec X_{lm} (\theta, \phi) \right] $$

$$g(\vec x) = \sum_{l=0}^{\infty} \sum_{m=-l}^{l} g_{lm}(r) Y_{lm}(\theta, \phi)$$

where $Y_{lm}$ are the usual spherical harmonics (see Jackson 3.53), $\vec Y_{lm}$ is simply $\hat r Y_{lm}$  and $\vec X_{lm}$ is defined as,

$$\vec X_{lm}(\theta, \phi) = \frac {-i}{\sqrt{l(l+1)}} \vec r \times \left( \nabla Y_{lm}(\theta, \phi) \right) $$

The three vector spherical harmonics $(\vec Y_{lm}, \vec X_{lm}, \hat r \times \vec X_{lm} )$ obey a few relations compiled below for ease of reference:

$$\nabla  \cdot \left( f(r) \left( \hat r \times \vec X_{lm} \right) \right) = - i \sqrt {l(l+1)} \frac {f(r)} {r} Y_{lm}$$

$$\nabla \cdot \left( f(r) \vec X_{lm} \right) = 0 $$

$$ \nabla \cdot \left( f(r) \vec Y_{lm} \right) = \frac {1}{r^2} \frac d {dr} \left(r^2 f(r) \right) Y_{lm}$$

$$\nabla \times \left( f(r) \left( \hat r \times \vec X_{lm} \right) \right) = \frac {-1} {r} \frac d {dr} \left( r f(r) \right) \vec X_{lm}$$

$$\nabla \times \left( f(r) \vec X_{lm} \right) = \frac {i \sqrt{l(l+1)} } {r} f(r) \vec Y_{lm} + \frac 1 r \frac d {dr} \left( r f(r) \right) \left( \hat r \times \vec X_{lm} \right)$$

$$\nabla \times \left( f(r) \vec Y_{lm} \right) = \frac {- f(r)} r i \sqrt {l (l+1)} \vec X_{lm}$$

$$ \vec X_{lm} \cdot ( \hat r \times \vec X_{lm}) = \vec Y_{lm} \cdot  ( \hat r \times \vec X_{lm}) = \vec Y_{lm} \cdot \vec X_{lm} = 0$$ 

$$ \int \vec X_{lm} \cdot \vec X_{l' m'}^* ~ d \Omega  = \int \vec Y_{lm} \cdot \vec Y_{l' m'}^* ~ d \Omega  = \int ( \hat r \times \vec X_{lm}) \cdot ( \hat r \times \vec X_{l' m'}^*) ~ d \Omega  = \delta_{ll'} ~ \delta_{m m'} $$

$$\int \vec Y_{lm} \cdot \vec X_{l' m'}^* ~ d \Omega = \int \vec Y_{lm} \cdot ( \hat r \times \vec X_{l' m'}^* ) ~ d \Omega  =  \int \vec X_{lm} \cdot ( \hat r \times \vec X_{l' m'}^* ) ~ d \Omega   = 0 $$

\newpage
\subsection{Newton's gravity}

Skipping this sub-section is totally okay, but for completeness and as a way to introduce some notation, let's actually go ahead and derive that the gravitational field follows the inverse square law. 

For a static mass distribution, we have:

$$\rho = m \delta^3_x  $$

where I am purposefully being a little bit non-commital about the delta function. In this present paper, it doesn't matter so much, but I think it's best to leave it a little open for future work. With that said, we can still say that outside of some radius R, it looks like a pure delta function... i.e.,

$$\delta^3_x = \delta(\vec x) ~~~ \textrm{if} ~~~ r > R $$

Using the irrotational part of the Maxwell equations:

$$\nabla \cdot \vec{E}_I = -4 \pi G \rho ~~~~~~~~~~ \nabla \times \vec{E}_I = 0$$

and using the spherical vector harmonics we developed elsewhere (actually the section above in this version of the paper).  We first write $\vec E_I $ as follows:

$$\vec E_I =  \sum_{l,m} \left[a_{lm}\vec Y_{lm} +b_{lm} (\hat r \times \vec X_{lm}) +c_{lm}\vec X_{lm} \right] $$

We re-write the the curl equation (there's a better way to say this)

$$ 0 = \nabla \times \vec E_I $$



Following Berrara [link](http://iopscience.iop.org/article/10.1088/0143-0807/6/4/014/meta) we assume that any vector field, $\vec V$, and any scalar, $g$, can be expanded as follows:

$$\vec V (\vec x) = \sum_{l=0}^{\infty} \sum_{m=-l}^{l} \left[a_{lm}(r) \vec Y_{lm}(\theta, \phi) +b_{lm}(r) \hat r \times \vec X_{lm} (\theta, \phi) +c_{lm}(r) \vec X_{lm} (\theta, \phi) \right] $$

$$g(\vec x) = \sum_{l=0}^{\infty} \sum_{m=-l}^{l} g_{lm}(r) Y_{lm}(\theta, \phi)$$

where $Y_{lm}$ are the usual spherical harmonics (see Jackson 3.53), $\vec Y_{lm}$ is simply $\hat r Y_{lm}$  and $\vec X_{lm}$ is defined as,

$$\vec X_{lm}(\theta, \phi) = \frac {-i}{\sqrt{l(l+1)}} \vec r \times \left( \nabla Y_{lm}(\theta, \phi) \right) $$

Using these definitions and after some slightly painful algebra, we can write out the final solution:

$$\vec E_I(\vec x) = \sum_{l,m} \left[ -\frac {4\pi i G }{2l+1} \frac {q_{lm}}{r^{l+2}} \sqrt {l(l+1)} \left(\hat r \times \vec X_{lm} \right)  - \frac {4 \pi G (l+1)}{2l+1} \frac {q_{lm}}{r^{l+2}} \vec Y_{lm} \right] $$

where

$$q_{lm} \equiv \int Y_{lm}^* (\theta, \phi) r^l \rho (\vec x ) d^3x = \int \rho_{lm} (r) r^{l+2}dr$$

Clearly then, if

$$\rho = m\delta(\vec x) $$

We get

$$\vec E_I(\vec x) = - \frac {mG} {r^2}  ~~~~~~~~~~~ \textrm{if} ~~~ r > R$$

as expected.
\newpage
\subsection{Radiation}
The constant will be discovered (don't know the right word for this) in subsection: Collapse and Stability
\newpage

\section{Stochastic Electrodynamics}
\subsection{Stochastic Electrodynamics}
\newpage

\section{Massive Particle}
\subsection{model}
\newpage
\subsection{Collapse and stability}
\newpage


\section{Three classical Experiments}
\subsection{Precession of the perehilion of Mercury}
For moving distributions, we need to take into account that a gravitational-magnetic field is introduced via

$$\vec B = \vec \beta \times \vec E ~~~~~~~~~~~ (15)$$

See the text around Jackson 11.150 for more information. 

Using the small-velocity approximation 

$$\vec B \approx \vec v \times \vec E ~~~~~~~~~~~ (16)$$

we can find the same equations as Einstein used to describe the precession of the perihelion of Mercury's orbit. 

To be precise, according to Sean M. Carroll's book "An Introduction to General Relativity: Spacetime and Geometry", we can write:

$$\frac 1 2 \left( \frac {dr}{d\lambda} \right)^2 + V(r) = \mathcal{E} ~~~~~~~~ \textrm{(Carrol 5.65)} $$

where

$$V(r) = \frac 1 2 \epsilon - \epsilon \frac {GM} r + \frac {L^2}{2r^2} - \frac {GML^2}{r^3} ~~~~~~~~ \textrm{(Carrol 5.66)} $$

According to Carroll: "A similar analysis of orbits in Newtonian gravity would have produced a similar result; the general equation (5.65) would have been the same, but the effective potential (5.66) would not have had the last term. (Note that this equation is not a power series in $1/r$, it is exact.)"

However, in our present theory, not only can we derive the last term (the term proprtional to $1/r^3$), but we see that this is only the low-limit approximation. I am not an experimentalist, but I believe that this would be a ripe area to explore whether this theory can produce results more accurate than General Relativity.


\newpage
\subsection{Bending of light around a mass}
This section concerns the two classical test of General Relativity concerning the bending of light around a massive object and the blue-shift of light as it falls down towards a massive object.

Main Postulate / Hypothesis: Empty space is filled with a randomly-fluctuating zero-point energy and a massive object absorbs some of this energy, thereby preventing collapse of the mass. I therefore postulate that a massive object induces a “field” (or maybe a momentum-flow is a better way to say it? sink?) in the surrounding space and when a photon travels through this “field,” the momentum of the photon is shifted in an additive and linear manner – in accordance with the following equation: 

$$\Delta \vec p = \int_{path} \vec P ~ dl ~~~~~~~~~~~ (30)$$

where we integrate along the path of the photon and $\vec P$ is defined as

$$\vec P = - \hat r \frac {Gm\hbar}{r^2} ~~~~~~~~~~~ (31)$$

We imagine a photon that emanates from a distant source, passes close by a massive object, then continues beyond the object to an observer. The closest point while passing, i.e., the impact parameter, is at a distance of $b$ from the center of the object. For ease of calculation, we will say that both the point of emanation and the observer are infinitely far from the massive object.

\begin{center}
	\includegraphics[scale=0.4]{light-bending.png}
\end{center}


We assume that to a first-order approximation the trajectory of the photon is a straight line. To calculate its change in momentum, we integrate along the straight-line path of the photon:

$$\Delta \vec p = \int_{path} \vec P ~ dl = \int_{-\infty}^{\infty} - \frac {Gm\hbar}{r^3} \vec r ~ wdz $$

We divide the problem into two parts: the momentum change in the direction parallel to the direction of travel of the photon and the momentum change perpendicular to the direction of travel. In calculating the momentum change in the direction parallel to the direction of travel, we note that as the photon approaches and then recedes from the massive object, the change in momentum cancels out to leave zero net change in the parallel direction. We can easily calculate the change in momentum in the direction perpendicular to the direction of travel:

$$\Delta p_y = \int_{-\infty}^{\infty}  \frac {Gm\hbar}{(z^2 + b^2)^{3/2}} b ~ wdz $$

This yields:

$$\Delta p_y = \hbar w \frac {2Gm} b$$

We therefore get:

$$\theta \approx \sin \theta = \frac {\Delta p_y}{p_z} = \frac {2Gm} b$$

Which yields the deflection angle:

$$\alpha = \frac {4Gm} b$$

This is the same result as in the theory of General Relativity.

\newpage
\subsection{Light falling in a gravity well}

Using the same postulate, we can calculate the blue-shift of light. As a photon falls into a gravity well, the momentum and hence the energy of the photon are shifted. Because the energy of a photon is generally expressed as a frequency, we say that the frequency of the photon observed at infinity is changed compared to when it is observed in a gravitational field at a distance R from the center of the massive object. We use the following equation to express this change:

$$\hbar w' = \hbar w + \Delta p $$

To calculate the change in momentum, we again integrate along the path of the photon:

$$\Delta p = \int_{\infty}^R \hat r \cdot \vec P w dr= \int_{\infty}^R - \frac {Gm\hbar}{r^2} w dr = \hbar w \frac {Gm}R $$

This yields:

$$\hbar w' = \hbar w \left( 1 + \frac {Gm}R \right) $$

At lowest order approximation, this is the same as that found in General Relativity

\newpage

\section{Dirac Equation}
\subsection{Dual transform}
\newpage
\subsection{Spinor Transformation}
\newpage

\end{document}
